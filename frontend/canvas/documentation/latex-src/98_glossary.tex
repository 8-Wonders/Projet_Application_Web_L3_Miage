\renewcommand{\sectionmark}[1]{\markboth{#1}{}}

\section*{Glossaire}
\addcontentsline{toc}{section}{Glossaire}
\label{sec:glossaire}
\sectionmark{Glossaire et Abréviations}

\noindent 
\textit{Abréviations}

\vspace{6pt}

\noindent 
\begin{tabular}{@{}p{4cm}l}
	CI&Intégration Continue (Continuous Integration)\\
	I18N/L10N&Internationalisation / Localisation\\
	IT&Test d'Intégration (Integration Test)\\
	JSON&JavaScript Object Notation\\
	PR&Pull Request (Demande de fusion)\\
	UT&Test Unitaire (Unit Test)\\
\end{tabular}

\vspace{18pt}

\noindent 
\textit{Technologies et Outils}

\vspace{6pt}

\noindent 
\begin{tabular}{@{}p{4cm}l}
	Git Flow&Modèle de gestion des branches pour Git\\
	GitHub&Plateforme d'hébergement de code et de gestion de projet\\
	GitHub Actions&Outil d'intégration continue intégré à GitHub\\
	JaCoCo&Outil de mesure de la couverture de code pour Java\\
	JavaDoc&Générateur de documentation pour le code source Java\\
	JUnit&Framework de test unitaire pour Java\\
	Maven&Outil de gestion de projet et d'automatisation de build\\
	Mockito&Framework de création d'objets simulés (mocks) pour les tests\\
	Spotless&Plugin Maven pour le formatage automatisé du code\\
\end{tabular}

\vspace{18pt}

\noindent 
\textit{Méthodologie}

\vspace{6pt}

\noindent 
\begin{tabular}{@{}p{4cm}l}
	Agile&Méthodologie de développement logiciel itérative\\
	Git Flow&Modèle de gestion des branches pour Git\\
	GRASP&Principes de conception (appliqués au projet)\\
	SOLID&Principes de conception (appliqués au projet)\\
	Strategy&Patron de conception comportemental (utilisé pour les bots)\\
	User Story&Description d'une fonctionnalité du point de vue de l'utilisateur\\
\end{tabular}

\vspace{18pt}

\noindent 
\textit{Termes Spécifiques au Projet}

\vspace{6pt}

\noindent 
\begin{tabular}{@{}p{4cm}l}
	*IT.java&Suffixe de fichier pour les tests d'intégration\\
	*Test.java&Suffixe de fichier pour les tests unitaires\\
	Âge I&Première phase du jeu 7 Wonders\\
	Adaptative&Nom d'une des stratégies sophistiquées de bot\\
	Bank&Classe gérant les ressources monétaires\\
	Bot&Classe représentant un joueur automatisé (IA)\\
	Config&Classe gérant la configuration\\
	MinMax&Nom d'une des stratégies sophistiquées de bot\\
	MAR&Nom d'une des stratégies sophistiquées de bot\\
	Player&Classe encapsulant l'état d'un joueur\\
	Session&Classe agissant comme le moteur central du jeu\\
	Strategy&Interface définissant le comportement d'un bot\\
\end{tabular}