\section{Working with LaTeX - EXAMPLE SECTION ON HOW TO USE LATEX}
\label{sec:latex}

Here I am writing something that I still need to revise.

The easiest way to edit LaTeX documents is using so-called online LaTeX editors, such as Overleaf.
To work with LaTeX on a Windows computer, you need a distribution such as texlive.
As well as a LaTeX editor that is as powerful as possible, for example, TeXstudio. You can download and install both from the internet.
For Mac and Linux users, other programs may be of interest.
\subsection{References in the document}
\label{sec:latex-verweise}
Text with reference to \cref{tab:trr} or \cref{fig:trichter} on \cpageref{fig:trichter}. With the cleveref package, the prefixes are set automatically.
The reference to \cref{eq:cost_balance} also works with \lstinline|\cref|. The parentheses are also set automatically.
\begin{center}
\begin{tabular}{lll}
Command &	Output& 	Example output\\
\midrule
\lstinline|\cref{Label}|& 	Object/Type and number/value &\cref{sec:latex-verweise}\\
\lstinline|\crefrange{Label1}{Label2}|& 	Object/Type from to &	\crefrange{sec:latex-verweise}{sec:latex-literatur}\\
\lstinline|\cpageref{Label}|& 	Page number with the word Page &\cpageref{sec:latex-verweise}\\
\lstinline|\cpagerefrange{Label1}{Label2}|& 	Page range &	\cpagerefrange{sec:latex-verweise}{sec:latex-literatur}\\
\lstinline|\namecref{Label}|& 	Object/Type &\namecref{sec:latex-verweise}\\
\lstinline|\labelcref{Label}|& 	Number/Value &\labelcref{sec:latex-verweise}\\
\lstinline|\labelcpageref{Label}|&	Just the page number&\labelcpageref{sec:latex-verweise}\\
\end{tabular}
\end{center}

\subsection{Formulas}
\label{sec:latex-formeln}
LaTeX is excellently suited for displaying formulas.
With the amsmath package, LaTeX becomes even more powerful. Numerous complex representations can be easily implemented.
Some examples:

\subsubsection{Isentropic Efficiency}
\begin{align}
\dot W_\text{el} &= \dot m \cdot \Delta h_s \left(\eta_\text{m,el} \cdot \eta_s\right)^\alpha &
\alpha &= \left\{
\begin{array}{r l}
1 & \quad \text{Turbines}\\
-1 & \quad \text{Pumps, Compressors}
\end{array}
\right.
\end{align}

\subsubsection{Physical Exergy for an Ideal Gas}
\begin{equation}
\frac{e^\text{PH}}{c_\text{p}T_0}=\left[\frac{T}{T_0}-1-\ln\frac{T}{T_0}\right]+\ln\left(\frac{p}{p_0}\right)^{\frac{\left(\kappa-1\right)}{\kappa}}
\end{equation}

\subsubsection{Exergy Destruction during Isobaric Heat Transfer}
\begin{equation}
\dot E_\text{D}=\sum\limits_j\left[\int_\text{i}^\text{e}\left(1-\frac{T_0}{T}\right)\text{d}\dot Q\right]_j	
\label{eq:ed_he}
\end{equation}

\subsubsection{A Cost Balance}
\begin{equation}
\sum_i{\left(c_i\dot E_i\right)_k} + \underbrace{\frac{\left(CC_\ell+OMC_\ell\right) BMC_k}{\tau \sum_k{BMC_k}}}_{\dot Z_k=\dot Z_k^\text{CI}+\dot Z_k^\text{OM}}
=\sum_e{\left(c_e\dot E_e\right)_k} + c_{\text{w,}k} \dot W_k + c_{\text{q,}k}\dot E_{\text{q,}k}
\label{eq:cost_balance}
\end{equation}

\subsection{Numbers and Units}
\label{sec:latex-zahlen-einheiten}

With the siunitx package, numbers and units can be easily set in running text as well as in mathematical environments.
Commands or short codes exist for all SI units and their derived units; see package documentation
\begin{center}
	\begin{tabular}{ll}
		Command &	Output\\
		\midrule
		\lstinline|\num{3,5}| &\num{3,5} \\
		\lstinline|\si{\meter}| &\si{\meter}  \\
		\lstinline|\SI{3,5}{\meter}|
&\SI{3,5}{\meter}  \\
		\lstinline|\numlist{3;3,5;4,2}| &	\numlist{3;3,5;4,2}  \\
		\lstinline|\numrange{3,5}{4,2}| &\numrange{3,5}{4,2}  \\
		\lstinline|\SIlist{3;3,5;4,2}{\meter}| &\SIlist{3;3,5;4,2}{\meter}  \\
		\lstinline|\SIrange{3,5}{4,2}{\meter}|&\SIrange{3,5}{4,2}{\meter}\\
	\end{tabular}
\end{center}
  

\subsection{Chemical Formulas}
\label{sec:latex-chem-formeln}

Here, the powerful mhchem package should be used.
Simple reaction equations can be written directly.
\begin{equation}
\ce{CH4 + 2O2 -> 2H2O + CO2}
\end{equation}

But more complex relationships and combinations with formulas are also possible.
\begin{multline}
\label{eq:kohle_reaktion}
%\begin{split}
\overbrace{\left(\ce{$c^*\cdot$ C + $h^*\cdot$ H + $o^*\cdot$ O + $n^*\cdot$ N + $s^*\cdot$ S}\right)}^{m_{\text{waf}}=\SI{1}{\kg}} +
\ce{$\nu_{\ce{O2}}\cdot$ O2 -> }\\
\ce{$\nu_{\ce{CO2}}\cdot$ CO2 + $\nu_{\ce{H2O}}\cdot$ H2O_{(l)} + $\nu_{\ce{N2}}\cdot$ N2 + $\nu_{\ce{SO2}}\cdot$ SO2}
%\end{split}
\end{multline}

\subsection{Literature References}
\label{sec:latex-literatur}

With biblatex and biber, literature references can be created quickly and easily.
The options for biblatex are passed in the preamble. Biber is used as the backend.
The sources are entered in a *.bib file. Literature management programs such as JabRef (open source) or Citavi (commercial, license via TU Berlin) can also be used for this.
\subsubsection{An Example with Citation}

The mathematical description of the steady-state power plant simulation suitably corresponds to an implicit system of equations. Instead of $y=f(x)$,
\begin{equation}
F\left(x, y\right)=0
\end{equation}
is written.\footnote{cf.
\cite[p.\,147]{papula2014}} This may even be mandatory, because according to \cite[p.\,260]{westermann2011} "[the] defining equation is difficult or impossible to solve explicitly for $y=f(x)$." Even a simple material property polynomial like the representation of entropy at reference pressure according to Knacke et al.\,\cite{knacke1991}, 
\begin{equation}
s^0=\text{S}^++\text{a}\,\ln(T/\text{K})+\text{b}\,y-\frac{\text{c}}{2}\,y^{-2}+\frac{\text{d}}{2}\,y^2\quad\quad\text{with}\quad y=10^{-3}T/\text{K}
\label{eq:knacke_entropie}
\end{equation}
thus only dependent on temperature, cannot be solved analytically for $T$.
\subsubsection{Simple References to Literature Sources}

Zoder et al.~\cite{zoder2018} conclude in their scientific article that the use of exergy-based methods is helpful in the analysis of energy conversion plants.
Other authors also present interesting results in their sometimes extensive publications, cf.~\cite{baehr1979,clausius1850,gasparovic1969,keenan1932,lojewski_urban1989}. Whereby so-called \textit{lumped references} should be avoided.
Each reference must be acknowledged individually, and it must be described, what is adopted from this source or which approaches, ideas, findings, etc.
from it are worth mentioning.

Some authors have written books worth reading, including Baehr and Kabelac~\cite{baehr2012}, Szargut~\cite{szargut2007} or Moran et al.~\cite{moran2014}.
Not to forget Müller~\cite{mueller2001} and the current compilation by the IEA~\cite{weo2016}.
Besides books and scientific articles, it is also possible to refer to book contributions \cite{lojewski_urban1989}, conference proceedings \cite{spliethoff2010a}, contributions in conference proceedings \cite{fraas1974,fraas1975}, dissertations \cite{ruegg1945,gaggioli1961}, scientific reports \cite{gutstein1975,nas_3_10606_1968}, internet sources and much more
.

\subsubsection{Literature Reference with Page Number}

Müller notes the thermal equation of state according to van der Waals for this, see~\cite[p.~100]{mueller2001}.
